 
%%%%%%%%%%%%%%%%%%%%%%%%%%%%%%%%%%%%
\section{Secant varieties}
\label{classicalAG-section-secants}
% Author: Alessandro Oneto

\begin{definition}[Join of varieties]
\label{classicalAG-definition-join}
% Author: Alessandro Oneto
Let $X,Y \subseteq \mathbb{P}^N$ be two algebraic varieties. Consider the incidence correspondence 
\[
    \itJoin^\circ(X,Y) = \{(x,y,p) ~:~ x \neq y,~p \in \langle x,y \rangle\} \subset X \times Y \times \bbP^N.
\]
This is a quasi-projective variety and its Zariski closure is the {\it abstract join} $\itJoin(X,Y) = \overline{\itJoin^\circ(X,Y)}$ of $X$ and $Y$. Consider the projections 
\[
    \xymatrix{
        X \times Y \times \bbP^N \ar[d]^{\pi_1} \ar[r]^(0.65){\pi_2} & \bbP^N\\
        X \times Y & .
    }
\]
The \emph{join} of $X$ and $Y$ is the scheme-theoretic image 
\[
    j(X,Y) = \pi_2(\itJoin(X,Y))
\]
This can also be regarded as the the closure of the union of all possible lines joining a point of $X$ and a point of $Y$, i.e.,
\[
    j(X,Y) = \overline{ \bigcup_{\substack{x \in X, y \in Y \\ x \neq y}} \langle x,y \rangle} \subset \bbP^N.
    % \{ p \in \mathbb{P}^N : p \in \langle x , y \rangle \text{ for some $x \in X, y \in Y$}\}}
\]
The projection $\pi_1$ realizes, locally, $\itJoin(X,Y)$ as a $\bbP^1$-bundle over (an open subset) of $X \times Y$. In particular, if $X$ and $Y$ are irreducible, than $Join(X,Y)$ is irreducible and, consequently, $j(X,Y)$ is irreducible as well.

Given varieties $X_1 \vvirg X_s \subseteq \bbP^N$, one can define their join recursively
\[
    j( X_1 \vvirg X_s) = j( j(X_1 \vvirg X_{s-1}), X_s);
\]
geometrically this is 
\[
    j( X_1 \vvirg X_s) = \overline{\bigcup_{\substack{x_1 \in X_1,\ldots,x_s \in X_s \\ \{x_1,\ldots,x_s\} \text{ indipendent}}} \langle x_1,\ldots,x_s \rangle} \subset \bbP^N.
\]
\end{definition}

Secant varieties are obtained by consecutive joins of a variety with itself.
\begin{definition}[Secant variety]
\label{classicalAG-definition-secantvariety}
% Author: Alessandro Oneto
Let $X \subseteq \mathbb{P}^N$ be an algebraic variety. The {\it $s$-th secant variety} of $X$ is 
\[
\sigma_s(X) = \overline{\bigcup_{\substack{x_1,\ldots,x_s \in X \\ \{x_1,\ldots,x_s\} \text{ indipendent}}} \langle x_1,\ldots,x_s \rangle} \subset \bbP^N.
\]
Recursively, $\sigma_1(X) = X$ and $\sigma_s(X) = j(X,\sigma_{s-1}(X))$.

The abstract $s$-th secant variety of $X$ can be defined as the abstract join of $s$-copies of $X$ as defined above. However, if is often convenient to consider a symmetrized version. Let $X^{\cdot s} = X^{\times s} / \frakS_s$ be the symmetrized product of $s$ copies of $X$. Then, the {\it abstract $s$-th secant variety} of $X$ is 
\[
    \Sec_s(X) = \overline{\left\{(\{x_1,\ldots,x_s\},p) ~:~ \substack{\{x_1,\ldots,x_s\} \text{ independent} \\ p \in \langle x_1,\ldots,x_s \rangle}\right\}} \subset X^s \times \bbP^N.
\]
Then, $\sigma_s(X) = \pi_2(\Sec_s(X))$, where $\pi_2 : X^s \times \bbP^N \rightarrow \bbP^N$ is the projection on the last factor.
\end{definition}

%%%%%%%%%%%%%%%%%%%%%%%%%%%%%%%%%%%%
\section{Expected Dimension}
\label{classicalAG-section-expectedDimension}
% Author: Alessandro Oneto
A first question we might ask on an algebraic variety is its dimension. 

\begin{lemma}
\label{classicalAG-lemma-expecteddimension}
% Author: Alessandro Oneto
Let $X_1\vvirg X_s \subseteq \bbP^N$ be algebraic varieties. Then 
\[
    \dim j(X_1\vvirg X_s) \leq \min\{ N, \smallsum_{i=1}^s \dim(X_i) + s-1\}.
\]
In particular, for any $X \subseteq \bbP^N$ algebraic variety, then 
\[
    \dim \sigma_s(X) \leq \min\{ N , s\dim(X) + s - 1\}.
\]
\end{lemma}
\begin{proof}
As they can be obtained as projection, the join $j(X_1,\ldots,X_s)$ and the secant variety $\sigma_s(X)$ have dimension smaller than their abstract analogous.

The dimension of the abstract join $\itJoin(X_1,\ldots,X_s)$, as defined in \ref{classicalAG-section-secants}, easily follows from the fact that locally it can be interpreted as a $\bbP^{s-1}$-bundle over $X_1 \ttimes X_s$: therefore, 
    \[
        \dim \itJoin(X_1,\ldots,X_s) = \sum_{i=1}^s \dim(X_i) + s-1.
    \]
    In particular, $\dim \Sec_s(X) = s\dim(X) + s-1$.
\end{proof}
One may {\it expect} that the upper bound of \ref{classicalAG-lemma-expecteddimension} is the actual dimension. This leads the the following definition. 
\begin{definition}
\label{classicalAG-definition-expecteddimension}
% Author: Alessandro Oneto
    Let $X_1\vvirg X_s \subseteq \bbP^N$ be algebraic varieties. The {\it virtual dimension} of $j(X_1,\ldots,X_s)$ is the dimension of the abstract join. Its {\it expected dimension} is the minimum between the virtual dimension and the dimension of the ambient space. In other words,
    \begin{align*}
        \virdim j(X_1,\ldots,X_s) & = \sum_{i=1}^s \dim(X_i) + s-1; \\ 
        \expdim j(X_1,\ldots,X_s) & = \min \{N,\virdim j(X_1,\ldots,X_s)\}.
    \end{align*}
    In particular, if $X \subseteq \bbP^N$ is an algebraic variety, 
    \begin{align*}
        \virdim \sigma_s(X) & = s\dim(X) + s-1; \\ 
        \expdim \sigma_s(X) & = \min \{N,\virdim \sigma_r(X)\}.
    \end{align*}
By \ref{classicalAG-lemma-expecteddimension} the expected dimension is always an upper bound for the dimension of a secant variety. We say that the algebraic variety $X$ is \emph{$s$-defective} if $\dim \sigma_s(X) < \expdim\sigma_s(X)$. 
\end{definition}

%%%%%%%%%%%%%%%%%%%%%%%%%%%%%%%%%%%%
\section{Rank and Border Rank}
\label{classicalAG-section-rank}
% Author: Alessandro Oneto

One of the main motivations to study secant varieties in relation to tensors is because they allow for a geometric definition of {\it rank} with respect to any algebraic variety and, in particular, to the ones defined in \ref{introduction-section-decomposable_tensors}.

\begin{definition}
\label{classicalAG-definition-Xrank}
% Author: Alessandro Oneto
    Let $X \subseteq \bbP^N$ be an algebraic variety and let $p \in \bbP^N$. The \emph{$X$-rank} of $p$ is the smallest number of points of $X$ whose linear span contains $p$. In other words,
    \[
        \rank_X(p) = \min\{s ~:~ \exists x_1,\ldots,x_s \in X, ~ p \in \langle x_1,\ldots,x_s \rangle\}.
    \]
\end{definition}

If $X$ is one of the varieties of decomposable tensors defined in \ref{introduction-section-decomposable_tensors}, we obtain the following notions of ranks for tensors.

\begin{definition}[Tensor rank]
\label{classicalAG-definition-tensor_rank}
% Author: Alessandro Oneto
    The \emph{tensor rank} is the rank of a tensor $T \in V_1 \ootimes V_d$ with respect to the Segre variety $\Seg(V_1,\ldots,V_d)$ (see \ref{introduction-definition-Segre}); i.e., 
    \[
        \rank(T) = \min\left\{s ~:~ T = \sum_{i=1}^s v_{i,1}\ootimes v_{i,d}, ~ v_{i,j} \in V_j\right\}.
    \]
    \begin{example}
        In the case of matrices, that is when $d=2$, the notion of tensor rank coincides with the usual notion of rank of matrices.
    \end{example}
\end{definition}

\begin{definition}[Symmetric tensor rank]
\label{classicalAG-definition-symmetric_tensor_rank}
% Author: Alessandro Oneto
    The \emph{symmetric tensor rank} is the rank of a tensor $T \in S^d(V)$ with respect to the Veronese variety $\nu_d(\bbP V)$ (see \ref{introduction-definition-Veronese}); i.e., 
    \[
        \rank_d(T) = \min\left\{s ~:~ T = \sum_{i=1}^s v_{i}^{\otimes d}, ~ v_{i} \in V\right\}.
    \]
    By interpreting symmetric tensors as homogeneous polynomials, as explained in \ref{introduction-subsection-symmetric_tensors}, the symmetric tensor rank is often called \emph{Waring rank}: this is the smallest possible length of a decompositions of a homogeneous polynomial as sum of powers of linear forms.
\end{definition}  

\begin{definition}[Partially-symmetric tensor rank]
\label{classicalAG-definition-partially_symmetric_tensor_rank}
% Author: Alessandro Oneto
    The \emph{partially-symmetric tensor rank} is the rank of $T \in S^{d_1,\ldots,d_m}(V_1,\ldots,V_m)$ with respect to the Segre-Veronese variety $\nu_{d_1,\ldots,d_m}(V_1,\ldots,V_m)$ (see \ref{introduction-definition-SegreVeronese}); i.e., 
    \[
        \rank_{(d_1,\ldots,d_m)}(T) = \min\left\{s ~:~ T = \sum_{i=1}^s v_{i,1}^{d_1}\ootimes v_{i,m}^{d_m}, ~ v_{i,j} \in V_j\right\}.
    \]
\end{definition}

\begin{definition}[Skew-symmetric tensor rank]
\label{classicalAG-definition-skewsymmetric_tensor_rank}
% Author: Alessandro Oneto
    The \emph{skew-symmetric tensor rank} is the rank of $T \in \Lambda^dV$ with respect to the Grassmannian $\Gr_d(V)$ (see \ref{introduction-definition-Grassmannian}), in its Pl\"ucker embedding; i.e., 
    \[
        \rank_{\wedge}(T) = \min\left\{s ~:~ T = \sum_{i=1}^s v_{i,1}\wwedge v_{i,d}, ~ v_{i,j} \in V_j\right\}.
    \]
\end{definition}

Clearly, if a point $p \in \bbP^N$ has $X$-rank equal to $r$, then $p \in \sigma_r(X)$. The converse is not true in general. The followign example is classical.

\begin{example}[$X$-rank is not always lower semicontinuous]
\label{classicalAG-example-Xranksemicontinuous}
% Author: Alessandro Oneto
  Let $X = \nu_3(\bbP^1) \subset \bbP^3$ be the rational normal cubic. If we identify the ambient space with the projective space of degree-$3$ binary homogeneous polynomials $\bbP(S^3\Bbbk^2)$, then $X$ is the variety of cubes of binary linear forms. Consider $xy^2 \in S^3\Bbbk^2$. Then, 
  \[
     xy^2 = \lim_{t \to 0} \frac{1}{3t}\left( x^3 - (x-ty)^3\right),
\]
 in particular, $xy^2 \in \sigma_2(X)$ since it is the limit of points lying on secant lines. However, it is possible to check that $\rank_3(xy^2) = 3$. This example shows that the rank with respect to the Veronese variety is not always lower semicontinuous. 
\end{example}

The existence of examples as the one of \ref{example:Xranksemicontinuous} motivate the following definition.
\begin{definition}[Border rank]
% Author: Alessandro Oneto
    \label{classicalAG-definition-border_rank}
    Let $X \subseteq \bbP^N$ be an algebraic variety and let $p \in \bbP^N$. The \emph{border $X$-rank} $p$ is the smallest $s$ such that $p$ belongs to the $s$-th secant variety of $X$, that is 
    \[
        \brank_X(p) = \min\{s ~:~ p \in \sigma_s(X)\}.
    \]
\end{definition}
%% add historical remark on definition of border rank
%% add reference to Bini example on matrix multiplication 

\section{Dimension of secant varieties}
\label{classicalAG-section-dimension}

Throughout more than a century, a great deal of research was devoted in the study of dimension of secant varieties, and in the classification of defective varieties. A very first result in this direction is \ref{classicalAG-lemma-palatini} which appears already in \cite{Pal09}. 

\begin{lemma}[Palatini's Lemma]
\label{classicalAG-lemma-palatini}
Let $X \subseteq \bbP^N$ be an algebraic variety with $\codim X \geq 2$ and let $C \subseteq \bbP^N$ be a linearly non-degenerate algebraic curve. Then either $\dim \bbJ(X,C) = \dim X + 2$. In particular, all secant varieties of algebraic curves have the expected dimension.
\end{lemma}


A fundamental tool in the study of dimension of secant varieties is Terracini's Lemma, dating back to \cite{Ter11}.
\begin{lemma}[Terracini's Lemma]
\label{classicalAG-lemma-terracini}
Let $X, Y \subseteq \mathbb{P}^N$ be algebraic varieties and let $\bbJ(X,Y)$ be their join. Let $z \in \bbJ(X,Y)$ be a smooth point such that $z \in \langle x , y \rangle$ for smooth points $x \in X$ and $y \in Y$. Then
\[
T_{z} \bbJ(X,Y) = \langle T_x X , T_y Y\rangle.
\]
In particular 
\[
\dim \sigma_r(X) = \dim \langle T_{x_1} X \vvirg T_{x_r} X\rangle
\]
for generic points $x_1 \vvirg x_r \in X$.
\end{lemma}
This result offers a \emph{dual} point of view to the problem of determining the dimension of secant varieties via interpolation of fat points. Let $X$ be an algebraic variety and let $\calL$ be a (very ample) line bundle on $X$. Let $\phi_\calL : X \to \bbP H^0(X, \calL)^\vee$ be the embedding defined by $\calL$. 
\begin{proposition}
 \label{classicalAG-proposition-interpolationfatpoints}
 Let $X$ be an irreducible algebraic variety, $\calL$ a very ample line bundle on $X$. Then
 \[
 \dim \sigma_r( \phi_\calL(X)) = h^0(\calL)-1 - h^0( \calI_{Z} \otimes \calL)
 \]
 where $Z$ is the union of $r$ double points supported at generic points of $X$.
\end{proposition}




